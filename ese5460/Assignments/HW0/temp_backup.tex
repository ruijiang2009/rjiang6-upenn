The Core Idea
At an optimal point (stationary point) of a constrained optimization problem, the statement says:

$$\nabla f = \lambda_1 \nabla g_1 + \lambda_2 \nabla g_2 + \cdots + \lambda_m \nabla g_m$$

Where:

$\nabla f$ = gradient of the objective function (loss)
$\nabla g_i$ = gradient of the $i$-th constraint
$\lambda_i$ = Lagrange multipliers (the coefficients)
What Does This Mean?
1. Geometric Interpretation:

The gradient of the loss function is parallel to (or lies in the span of) the constraint gradients
You cannot improve the objective by moving along the constraint surface
The gradient $\nabla f$ is "blocked" by the constraints
2. From Your Lecture Notes:

The notes explain this beautifully:

"At the minimum, the level curves are tangent to each other, so the normal vectors $\nabla f$ and $\nabla g$ are parallel."

And more generally:

"Why the method works: at constrained min/max, moving in any direction along the constraint surface $g = c$ should give $df/ds = 0$. So, for any $\hat{u}$ tangent to ${g = c}$, $\frac{df}{ds}\big|_{\hat{u}} = \nabla f \cdot \hat{u} = 0$, i.e. $\hat{u} \perp \nabla f$. Therefore $\nabla f$ is normal to tangent plane to $g = c$, and so is $\nabla g$, hence the gradient vectors are parallel."

Example from Your Notes
For the problem: minimize $f(x,y) = x^2 + y^2$ subject to $xy = 3$

At the optimum:

$\nabla f = (2x, 2y)$
$\nabla g = (y, x)$
The condition becomes: $(2x, 2y) = \lambda(y, x)$
This gives:

$2x = \lambda y$
$2y = \lambda x$
The Lagrange multiplier $\lambda$ is the coefficient that makes $\nabla f$ equal to a scalar multiple of $\nabla g$.

Why "Linear Combination"?
When you have multiple constraints $g_1 = c_1, g_2 = c_2, \ldots, g_m = c_m$:

$$\nabla f = \lambda_1 \nabla g_1 + \lambda_2 \nabla g_2 + \cdots + \lambda_m \nabla g_m$$

The gradient of the loss is a weighted sum (linear combination) of all constraint gradients, where the weights are the Lagrange multipliers.

Physical Intuition
Think of it this way:

You want to minimize $f$, so naturally you'd move in the direction of $-\nabla f$
But the constraints "push back" with forces proportional to $\nabla g_i$
At equilibrium (the optimal point), these forces balance: $\nabla f$ equals the combined effect of all constraint forces
The Lagrange multipliers $\lambda_i$ tell you how strongly each constraint is "pushing" against your objective at the optimal point.

