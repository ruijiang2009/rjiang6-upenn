\documentclass[11pt, reqno, letterpaper, twoside]{amsart}
\linespread{1.2}
\usepackage[margin=1.25in]{geometry}

\usepackage{amssymb, bm, mathtools,physics}
\usepackage[usenames,dvipsnames,svgnames,table]{xcolor}
\usepackage[pdftex, xetex]{graphicx}
\usepackage{enumerate, setspace}
\usepackage{float, colortbl, tabularx, longtable, multirow, subcaption, environ, wrapfig, textcomp, booktabs}
\usepackage{pgf, tikz, framed}
\usepackage[normalem]{ulem}
\usetikzlibrary{arrows,positioning,automata,shadows,fit,shapes}
\usepackage[english]{babel}

\usepackage[final]{microtype}

\theoremstyle{plain}
\newtheorem{theorem}{Theorem}

\theoremstyle{definition}
\newtheorem{solution}[theorem]{Solution}

\usepackage{times}
\title{ESE 546, Fall 2022\\[0.1in]
Homework 1}
\author{
Rui Jiang [rjiang6@seas.upenn.edu]
}

\begin{document}
\maketitle

\begin{solution}[Time spent: 1 hour]
~\\
\textbf{Part (a):}
The slack variable formulation allows some data points to violate the margin constraint.
To minimize these violations while still maximizing the margin, we add a penalty term to 
the objective function. 

One common formulation is:
\begin{align*}
\text{minimize} \quad & \frac{1}{2}\|\theta\|^2 + \frac{1}{n} \sum_{i=1}^{n} \xi_i \\
\text{subject to} \quad & y_i(\theta^T x_i + \theta_0) \geq 1 - \xi_i, \quad \forall i = 1, \ldots, n \\
& \xi_i \geq 0, \quad \forall i = 1, \ldots, n
\end{align*}
where $\frac{1}{n} \sum_{i=1}^{n} \xi_i$ is a normalized penalty by the number of samples.
\end{solution}

\clearpage
\begin{solution}[Time spent: 1 hour]
Your solution goes here.
\end{solution}

\clearpage
\begin{solution}[Time spent: 1 hour]
Your solution goes here.
\end{solution}

\end{document}